\documentclass[Aluffi.tex]{subfiles}

\begin{document}

\section{II: Rings and Modules}

\subsection{Section 5}

\begin{problem}
Let $R$  be a commutative ring, viewed as a $R$-module over itself and let $M$ be an $R$-module.
Prove that $Hom_{R_{Mod}}(R.M) \cong M$ as an isomorphism of $R$-modules.
\end{problem}

\begin{proof}
Consider the map $\phi: Hom_{R_{Mod}}(R,M) \rightarrow M$ that takes $\psi_m$ which is a $R$-module homomorphism from $R \rightarrow M$ such that $\psi_m(1_R)= m$ where $1_R$ is the multiplicative unit of $R$
to $m$ itself. We see that such a map is indeed a homomorphism as $\phi(\psi_m   \circ \psi_n)= \phi(\psi_{mn}) = mn = \phi(\psi_m)\phi(\psi_n)$. Injectivity follows from our definition as we map the $R$'s unit to $m$. This induces an unique $R$-module homomorphism. Surjectivity follows as well as any such $\psi
_{m} \in Hom_{R_{Mod}}(R,M)$. Thus, as $\phi$ is a bijective homomorphism, it follows that it is a $R$-module isomorphism.
\end{proof}

\begin{problem}
Let $R$ be a commutative ring and let $M$ be an $R$-module. Prove that the operation of composition on the $R$-module $End_{R_{Mod}}(M)$ makes the latter into an $R$-algebra in a natural way.
\end{problem}

\begin{proof}
Given the $R$-module $End_{R_{Mod}}(M)$, we shall prove that the multiplicative operation of composition (which turns $End_{R_{Mod}}(M)$ into a ring), turns it into an $R$-algebra. Given any $\phi \in End_{R_{Mod}}(M)$. For any given $\phi_1,\phi_2 \in End_{R_{Mod}}(M), m \in M$ $a \in R$, then  $(r\phi_1)(m) \circ \phi_)(m)  = (r\phi_1)(a\phi_2(m)) = r\phi_1a(\phi_2(m)) = ra(\phi_1(m) \circ \phi_2(m))$ (???). (consider acrion of $r$ on composition to get equality)
\end{proof}

\begin{problem}
Let $R$ be an integral domain and let $I$ be a nonzero principal domain of $R$. Prove that $I$ is isomorphic to $R$ as an $R$-module.
\end{problem}

\begin{proof}
Let $I = (i)$ be our principal ideal in $R$. Consider the map $\phi: R \rightarrow I$ as $r \mapsto ri$. This map can be easily shown to be an abelian group homomorphism. Also for $s \in R$, $\phi(sr) = sri = s(ri) = s \phi(r)$. Thus, $\phi$ is also a $R$-module homomorphism. Injectivity of our map follows from the assumption that $R$ is an integral domain and $i \neq 0$. Surjectivity follows as $\{ri \vert r \in R\} = I$. Thus, $\phi$ is a $R$-module isomorphism. 
\end{proof}

\begin{problem}
Let $R$ be a commutative ring. $M$ be a $R$-module, and let $a \in R$ be a nilpotent element, determining a submodule $aM$ of $M$. Prove that $M = 0$ iff $aM = M$. (This is a special case of Nakayama's Lemma)
\end{problem}

\begin{proof}
If $M = 0$, then it trivially implies that $aM =M$. Conversely, let $M \neq 0$ and $aM = M$, then there exists  nonzero elements $m,m_1 \in M$ such that $m = am_1$. By the same argument, there exists nonzero $m_2$
such that $m_1 = am_2$. Continue this procedure for $n$ steps where $a^n = 0$. Then we have that $m = a^nm_n = 0$, contradicting our assumption above. Hence, $M = 0$ exactly.
\end{proof}

\begin{problem}
Let $R$ be a commutatative ring and let $I$ be an ideal of $R$. Noting that $I^j \cdot I^k \subseteq I^{jk}$,
define a ring structure on the direct sum 
$$\text{Rees}_R(I) = \bigoplus_{j \geq 0} I^j = R \oplus I \oplus I^2 \oplus ...$$
By embedding $R$ into the first term, we create an $R$-algebra, called the Rees Algebra of $I$. Prove that if $a \in R$ is a nonzero divisor, then $Rees_{R}((a)) \cong R[x]$ as an $R$-algebra isomorphism. 
\end{problem}

\begin{proof}
Consider the map $F: \text{Rees}_R((a)) \rightarrow R[x]$ that takes $s_i = (0,0,0,...,1_Ra^i,...) \mapsto x^i$ for $i \geq 0$.  Consider the definition of the Rees and algebra and note that 
$ I = \{ri \vert i \in I\}$, $I^2 = \{\sum_{r,t \in R,finite} ra \cdot ta \}$ and hence forth. This exactly concides with multiplication operation on elements in the ring $R[X]$. Hence, the map is a surjective ring homomorphism. Injectivity follows from the assumption that $a$ is a non-zero divisor in $R$. The map preserves the $R$-module structure and thus is an $R$-algebra isomorphism.  
\end{proof}


\subsection{Section 6}

\begin{problem}
Prove that $\mathbb{Z}^{\mathbb{N}} \not\cong \mathbb{Z}^{\oplus \mathbb{N}}$
\end{problem}

\begin{proof}
We note that $\mathbb{Z}^{\oplus \mathbb{N}} = \{ \alpha: A \rightarrow \mathbb{Z} \vert \alpha(i) \neq 0 \text{for finite number of $i$} \}$ which has cardinality $\aleph_1 > |\mathbb{Z}^{\mathbb{N}}|$ 
\end{proof}


\begin{problem}
Let $R$ be a ring. A $R$-module is cyclic if $M = \langle m \rangle$ for some $m \in M$. Prove that simple modules are cyclic. Prove that an $R$-module $M$ is cyclic if and only if $M \cong R/I$ for some left ideal $I$ Prove that every quotient of a cyclic module is cyclic. 
\end{problem}

\begin{proof}
Let $\langle m \rangle$ be the cyclic group generated by some non-zero $m \in M$. As $M$ is an abelian group, $\langle m \rangle$ is a normal subgroup. $M / \langle m \rangle$ is a quotient ableian group, but cannot be a $R$-module quotient. Let $\phi: M \rightarrow M / {\langle m \rangle}$ be the canonical map on its factor group.
The kernel of $\phi$ must be a submodule of $M$ which only can be $0$ or $M$ Since we set $m$ be the non-zero, the only choice can be $ker(\phi) = M$ which directly yields that $M$ is cyclic. \par
Let $M = \langle m \rangle$ be a cyclic module and let $M/N$ be its quotient module. Let $mN$ be the coset where $m$ resides. Consider $(mN)^2$. Then $m^2 \in (mH)^2$.  Expanding on this argument, $m^2 \not\in mN$. 
since $m^2m^{-1} = m \in N$ Hence $\langle x \rangle \subseteq N$, which makes the quotient cyclic.
If $M/N$ is a non-trival quotient $R$- module.  

(Rest of problems are yet to be typed.) 
\end{proof}

\subsection{Section 7}

\begin{problem}
Suppose the seqeuence $... \rightarrow 0 \xrightarrow{\phi_1}  M  \xrightarrow{\phi_2} 0 \rightarrow ....$
is exact. Prove that $M\cong 0$
\end{problem}

\begin{proof}
By exactness, we see that $\phi_1$ induced $\phi_2$ into a monomorphism. Thus, $M \cong 0$. 
\end{proof}

\begin{problem}
Assume that the complex $... \rightarrow 0 \rightarrow M \rightarrow M' \rightarrow 0 \rightarrow ...$
\end{problem}

\begin{proof}
Simple application of the first isomorphism theorem yields our desired result.
\end{proof}


\begin{problem}
Construct short exact seqeunces of the follows:

$$ 0 \rightarrow \mathbb{Z}^{\oplus \mathbb{N}} \rightarrow  \mathbb{Z}^{\oplus \mathbb{N}} \rightarrow \mathbb{Z} \rightarrow 0 $$
\end{problem}

\begin{proof}
Consider the map $\phi_1: \mathbb{Z}^{\oplus \mathbb{N}} \rightarrow \mathbb{Z}^{\oplus \mathbb{N}}$ such that $(z_1,z_2,z_3,...) \mapsto (0,z_1.z_2,z_3,...)$. The map $\phi_2 \mathbb{Z}^{\oplus \mathbb{N}} \rightarrow \mathbb{Z}$ will be a $\mathbb{Z}$-module homomorphism that makes the cosets $[z]$ injectivity to $z$. Note that we can construct a more general short seqeuence:

$$ 0 \rightarrow \mathbb{Z}^{\oplus \mathbb{N}} \rightarrow  \mathbb{Z}^{\oplus \mathbb{N}} \rightarrow \mathbb{Z}^{\oplus \mathbb{N}} \rightarrow  0$$ by shifting $\phi_!$ move by $n \in \mathbb{N}$ steps.  
\end{proof}


\begin{problem}
In the snake lemma, assume that $\lambda$ and $\nu$are isomorphisms. Use the sanke lemma and prove that $\mu$ is an isomorphism. (Short Five Lemma)
\end{problem}

\begin{proof}
If $\lambda$, and $\nu$ are isomorphisms, then exact sequence: 

$$0 \rightarrow ker\lambda \rightarrow ker\mu \rightarrow ker \nu \rightarrow 0 $$ becomes
$$ 0 \rightarrow 0 \rightarrow ker \nu \rightarrow 0  $$

Thus $ker \mu = 0$. We can apply the same argument to the exact sequence of cokernels on the bottom row to yield that $coker (\mu) = 0$. Hence, $\mu$ is an isomorphism.
\end{proof}

\end{document}



