\documentclass[12pt]{article}
\setlength{\oddsidemargin}{0in}
\setlength{\evensidemargin}{0in}
\setlength{\textwidth}{6.5in}
\setlength{\parindent}{0in}
\setlength{\parskip}{\baselineskip}

\usepackage{amsmath,amsfonts,amssymb,amsthm}
\usepackage{subfiles}

\title{Problems in Alperin and Bell's Groups and Reps}
\author{Edward Kim}

\begin{document}
\maketitle
\clearpage

\section{Chapter 1 Section 2 (Automorphisms)}

Let $N$ and $H$ be groups. We deem a group $E$ to be an extension of $N$ by $H$ if 

$$ 0 \rightarrow N \xrightarrow{i} E \xrightarrow{\pi} H \rightarrow 0 $$ is an exact sequence ($i$ is a monomorphism and $\pi$ is an epimorphism).

An extension $E$ is called a split extension if there exists a homomorphism $t: H \rightarrow E$ such that 
$\pi \circ t$ yields the identity map on $H$. $t$ is deemed as the splitting homomorphism. Show that $E$ is a split extension iff it is exactly $N \rtimes_{\phi} H$ where the conjugation homomorphism $\phi: H \rightarrow Aut(N)$. 

\begin{proof}
Suppose $E = N \rtimes H$, then defining the splitting homomorphism as $t(h) = (1,h)$. It follows that $\pi \circ t = id_H$. We note that $i(N) = ker(\pi)$ as $\pi((N,1)) = 1_H$. Conversely, let $E$ be a split extension and $t$ be the splitting homomorphism for $E$. Define the map $\psi: E \rightarrow N \rtimes H$ : for all $n \in N$, $\psi(i(n)) = (n,1)$ and for all $h \in H$, $\psi(t(h)) = (1,h)$. To show that $\psi$ is indeed a group homomorphism, observe that, without loss of generality, for $x,y \in t(H)$ where $x \in i(N), y\not\in i(N)$, we can define 

$$\psi(xy) = \psi(i(n)t(h)) = (n,1) \times (1,h) = (n,h)  $$ Since the sequence above is exact, $\psi$ is injective and by $t$, the map is also surjective. Thus, we see that $E$ is isomorphic to a semidirect product of $N$ and $H$.  
\end{proof}  

If $E$ is an extension of $N$ by $H$, we generally cannot expect to find a splitting homomorphism $t H \rightarrow E$ such that $E$ splits. Howeveer, since $H \cong E/N$, we can create a set mapping section $t: H \rightarrow E$ such that its image is a traversal for $N$. If $t(1_H) = 1_E$ then we call the section normalized.

Now let $t$ be a normalized section of an extension $E$. Let $\Psi: E \rightarrow Aut(E)$ be the homomorphism that sends $e$ to its inner automorphism. Since the image of $N$ under inclusion is normal in $E$, we can consider each $\Psi(x)$ to be an automorphism of $N$. Let $f: H \times H \rightarrow N$ and $\phi: H \rightarrow Aut(N)$ be defined as follows:

$$f(\alpha,\beta) = t(\alpha)t(\beta)t(\alpha\beta)^{-1} $$ 
$$ \phi(h) = \Psi(t(h))$$

We call $(f,\phi)$ as a factor pair arising from $t$. The factor pair has the following properties:

\begin{enumerate}
\item $f(\alpha,1) = f(1,\alpha) = 1$ for every $\alpha \in H$ and $\phi(1)$ is the trivial automorphism.
	\begin{proof}
	$F(\alpha,1) = t(\alpha)t(1)t(\alpha)^{-1} = 1 = t(1)t(\alpha)t(\alpha)^{-1} = f(1,\alpha)$ and $\psi(1) = \Psi(t(1)) = id_{Aut(N)}$.
	\end{proof}
\item $\phi(\alpha)\phi(\beta) = \Psi(f(\alpha,\beta)) \phi(\alpha\beta), \alpha,\beta \in H$

	\begin{proof}
	$$\Psi(f(\alpha,\beta)) \phi(\alpha\beta) = \Psi(f(\alpha,\beta)) \circ \Psi(t(\alpha\beta)) = \Psi(f(\alpha,\beta) \cdot t(\alpha\beta))  = \Psi(t(\alpha)t(\beta)t(\alpha\beta)^{-1}\cdot t(\alpha\beta))$$
	$$\Psi(t(\alpha)) \circ \Psi(t(\beta)) = \phi(\alpha)\phi(\beta) $$
	\end{proof}
	
\item $f(\alpha,\beta)f(\alpha\beta,\gamma) = \phi(\alpha)(f(\beta,\gamma))f(\alpha,\beta\gamma), \alpha,\beta,\gamma \in H$	

\begin{proof}
$$f(\alpha,\beta)f(\alpha\beta,\gamma) = t(\alpha)t(\beta)t(\alpha\beta)^{-1}t(\alpha\beta)t(\gamma)t(\alpha(\beta\gamma)) = $$ $$t(a)(t(\beta)t(\gamma)t(\beta\gamma)^{-1)} t(a)^{-1}  (t(a)t(\beta\gamma)t(\alpha(\beta\gamma))^{-1}  = \phi(\alpha)(f(\beta,\gamma))f(\alpha,\beta\gamma)$$

\end{proof}

\end{enumerate}

Using this information, we can externalize the definition of an extension of two groups $N,H$ by endowing the following operation on $N \times H$:

$$(n_1,h_1)(n_2,h_2) = ((n_1)\phi(f(h_1,h_2))(n_2),h_1h_2) $$
   
where $f:H \times H \rightarrow N$ is a mapping such that the following three properties are observed
and $\phi: N \rightarrow Aut(N)$ by mapping each element to its corresponding inner automorphism.


\section{Chapter 1 Section 3 (Group Actions)}

\subsection{Problem 1}
Suppose that the finite simple group $G$ has order $r!$ and there exists a subgroup $H < G$ such that 
$|G : H | = r$. We note that by Lagrange's Theorem that:

$$ |G:H| = r = |G : N(H)| \cdot |N(H) : H|$$
 
Thus, $r$ divides $(|G : N(H)| \cdot |N(H) : H|)$. However, $|N(H)|$ must divide $r!$ if $|N(H)| = r!$, then
$H$ would be a normal subgroup, contradicting our simpleness assumtion. In other words, $|G:N(H)| > 1$. Thus, $|N(H)|$ must consist of factors of $(r-1)!$, and $r$ divides $|G:N(H)|$ iff $|N(H)| = (r-1)!$ which leaves $|N(H) : H|  = 1$ or $N(H) = H$. Let $\theta: G \rightarrow \Sigma_r$ be the homomorphism which is defined on the actions of elements $g$ on the cosets of $G/H$. It follows that $H \subset Ker(\theta) = K$. By a similar argument,

$$ r = |G:H| = |G:K|\cdot |K:H| $$ 
 $|G:K| > 1$ so $|G:K| = (r-1)!$ which gives that $|K : H| = 1$ or $K = H$, contradicting our simplicity assumption once again. We use a similar argument for orders $> r!$ for $(r! + e)$ for $e\leq r$, $|N(H)| = e$. However, $er/(r!+e) = |N(H) : H| < 1$.
 
 
\subsection{Problem 4}

Let $G$ be the subgroup of $\Sigma_5$ generated by the cycle $(1,2,3,4,5)$ and let $G$ act on $X = \{1,2,3,4,5 \}$ in the canonical manner. We shall prove that this action is primitive but not doubly transitive.

\begin{proof}
Let $A \subset X$ be a nontrivial subset. Note that the subgroup generated by our cycle $\sigma$ has order 5.
Let $x \in A$ and $y \in X/A$. Then $\sigma^{|x-y|}(x) = y$. Hence, the only feasible blocks would be the singletons $\{x\}, x \in X$ and $X$ itself. However, consider the pair $(2,3), (3,5)$. No element of the subgroup above will permute 2 to 3 and 3 to 5 in tandem since $\sigma^n$ shifts all elements $n$
 places 
 \end{proof}
 
Note that the fact that 5 is prime affects our results above. (Expand on this).

  
\end{document}