\documentclass[12pt]{article}
\setlength{\oddsidemargin}{0in}
\setlength{\evensidemargin}{0in}
\setlength{\textwidth}{6.5in}
\setlength{\parindent}{0in}
\setlength{\parskip}{\baselineskip}

\usepackage{amsmath,amsfonts,amssymb,amsthm}
\usepackage{subfiles}

\title{Problems in Atiyah's Commutative Algebra}
\author{Edward Kim}

\begin{document}
\maketitle
\clearpage

\section{Extensions and Contractions}

Let $A \rightarrow B$ be a ring homomorphism. For an ideal $\mathfrak{a}$,  $f(\mathfrak{a})$ is not an ideal in general.
Consider the embedding $\phi: \mathbb{Z} \rightarrow \mathbb{Q}$, and take $f((m))$ for any $m \in \mathbb{Z}$. In general, for any
$q \in \mathbb{Q}$, $qf(m) \not\in (f((m)))$. To complete the image, we take the
extension $\mathfrak{a}^e$ to be the ideal generated by $f(\mathfrak{a})$ in $B$. We define the contraction of an ideal $\mathfrak{b}^c$ in $B$ to be
$f^{-1}(B)$ which is always an ideal in $A$.

\subsection{Properties of Extensions and Contractions}

The following operations are closed under extensions and contractions. Let $\mathfrak{r(a)}$ denote the radical of ideal $\mathfrak{a}$. 
\begin{enumerate}
 \item $(\mathfrak{a}_1 + \mathfrak{a}_2)^e = Bf(\mathfrak{a}_1 + \mathfrak{a}_2) = Bf(\mathfrak{a}_1) + Bf(\mathfrak{a}_2) = \mathfrak{a}_1^e + \mathfrak{a}_2^e$
 \item $(\mathfrak{a}_1 \cap \mathfrak{a}_2)^e \subseteq \mathfrak{a}_1^e \cap \mathfrak{a}_2^e$. Easily follows since if $x \in (\mathfrak{a}_1 \cap \mathfrak{a}_2)^e$, then
 $x \in \mathfrak{a}_1^e$ and  $x \in \mathfrak{a}_2)^e$. Howeverm, the converse is generally not true. $Let \phi: \mathbb{Z} \rightarrow \mathbb{Q}$ be the natural embedding and let
 $\mathfrak{a} = (p_1)$ and $\mathfrak{b} = (p_2)$ for two primes $p_1,p_2$. Then $\phi(\mathfrak{a}) \cap \phi(\mathfrak{b})$ (How about the example with the map $\mathbb{Z} \rightarrow \mathbb{Z}[i]$ with
 $\mathfrak{a} = (2)$ and $\mathfrak{b} = (5)?)$
 \item $(\mathfrak{a}_1\mathfrak{a}_2)^e = Bf(\mathfrak{a}\mathfrak{b}) = Bf(\mathfrak{a})f(\mathfrak{b}) =Bf(\mathfrak{a})Bf(\mathfrak{b}) = \mathfrak{a}^e\mathfrak{b}^e$
 \item $(\mathfrak{a} : \mathfrak{b})^e \subseteq (\mathfrak{a}^e : \mathfrak{b}^e)$ ( Let $y \in (\mathfrak{a} : \mathfrak{b})$, then 
 $Bf(y)f(\mathfrak{b}) \subseteq f(\mathfrak{a}))$
 \item $\mathfrak{r(a)}^e \subseteq \mathfrak{r(a^e)}$. If $y \in r(\mathfrak{a})^e$, then $y = bx$ where $x \in \mathfrak{r(a)}$. Thus, $ y \in Bf(\mathfrak{r(a)})$ and $y \in \mathfrak{r(a)}^e$
 \end{enumerate}

 \section{Exercises}
 
 \subsection{Problem 1}
 \newpage
 Question: If $x$ is a nilpotent element in commutative ring $A$, show that $1 + x$ is a unit. Deduce that the sum of a nilpotent element and a unit is a unit.
 \begin{proof}
 Since $x \in \mathfrak{R}$ where $\mathfrak{R}$ is the nilradical, there exists $n > 0$ such that $x^n = 0$.
 Multiplying by $x^{n-1}$ yields $x^{n-1}(1 + x) = x^{n-1} + 0 = x^{n-1}$. Thus, $1 + x$ must be a unit. We can generalize this result as follows:
 Let $u$ be a unit in the ring $A$ and $x$ as defined above. Performing a similar operation as above, we receive
  $$ x^{n-1}(u + x) = ux^{n-1} $$. As $u$ is a unit, let $y \in R$ be chosen such that $yu = 1$. Multiplying $y$ on both sides gives us: $$ x^{n-1}(1 + yx) = x^{n-1} $$. As $yx \in \mathfrak{R}$, we see that the addition of a unit with a nilpotent element gives us a unit in $R$.
 \end{proof}

Problem 1 relates to the following results from Problem 10:

\subsection{Problem 10}

Question: Let $A$ be a ring and $\mathfrak{R}$ be its nilradical. Show that the following are equivalent:
\begin{enumerate}
 \item Show that $A$ has exactly one prime ideal.
 \item Every element of $A$ is either a unit or nilpotent.
 \item A / $\mathfrak{R}$ is a field.
\end{enumerate}

\begin{proof}
 For the implication $2 \implies 3$. under the assumption of 2, the non-trivial cosets of $A / \mathfrak{R}$ will all contain units.
 Thus, for any $\bar{u} \in A / \mathfrak{R}$, the multiplication of coset of $\bar{u^{-1}}$ will yield $\bar{1}$. Thus, $A / \mathfrak{R}$ is a field. \par
 For $3 \implies 1$, If $A / \mathfrak{R}$, then $\mathfrak{R}$ must be maximal. Thus, $\mathfrak{R}$ is prime. However as $\mathfrak{R}$ is the intersection of all prime ideals, it follows that it must be the only prime ideal.
 For $1 \implies 2$, by proposition 1.9, we know that any $x \in A - \mathfrak{R}$ must be a unit as if there is only one prime ideal, it must exactly be a maximal ideal. Thus, $A$ is only composed of units and nilpotent elements.
\end{proof}


Remark: This ring seems to be partitioned into two classes: one with elements that are inextricably linked to the muliplicative identity 1 and the other consisting of elements eventually return to 0.


 
\subsection{Problem 2}

Let $A$ be a commutative ring and $A[x]$ its corresponding polynomial ring with an indeterminate $x$ with coefficients in $A$.
\begin{enumerate}
 \item 
\end{enumerate}

\subsection{Problem 6}
Let $A$ be a commutative ring such that every ideal not contained in the nilradical contains a non-zero idempotent (element $e$ such that $e^2 = e \neq 0$. Prove that the nilradical and the Jacobson radical are equal.

\begin{proof}
 Let $\mathfrak{R}$ denote the nilradical and let $\mathfrak{J}$ denote the Jacobson radical. Since every maximal ideal is prime, it follows that $\mathfrak{R} \subseteq \mathfrak{J}$. Suppose for the sake of contradicion that the inclusion is proper, then by our assumption there exists an idempotent $e \in \mathfrak{J}$. Then, we know that $1-ey$ is a unit for all $y \in A$. Then the following equality holds: there exists $s \in R$ such that 
 $$(1-ey)s = 1 \implies e(1-ey)s = e \implies e(1-y)s = e$$
 Hence, $(1-y)s = 1$ and $(1-y)$ is a unit for all $y \in A$. So by the same proposition, $1 \in \mathfrak{J}$, a contradiction.
\end{proof}

\subsection{Problem 11} 
Let $A$ be a commutative ring. $A$ is deemed as Boolean if $x^2 = x$ for all $x \in A$. In a Boolean ring $A$, show that
\begin{enumerate}
 \item $2x = 0$
 \item every prime ideal $\mathfrak{p}$ is maximal, and $A/\mathfrak{p}$ is a field with two elements.
 \item every finitely generated ideal in $A$ is principal.
\end{enumerate}

\begin{proof}
 \begin{enumerate}
  \item $(2x)^2 = 2x \implies 4x^2 = 2x$. Since $x^2 = x$, the equality would only hold if $2x = 0$.
  \item If $\mathfrak{p}$ is a prime ideal in $A$, then $A / \mathfrak{p}$ is an integral domain. Since for $2x = 0$ and $x^2 = x$ for every $x \in A$, there can only be two equivalence classes, namely the equivalence classes of $x + \mathfrak{p} = 1 + \mathfrak{p}$ where $x \neq 0$. and $2(x + \mathfrak{p})= 0 + \mathfrak{p}$. This gives us the field of two elements with the inverses being the classes themselves. This directly gives us that $A / \mathfrak{p}$ is a field and hence every prime ideal is maximal. 
  \item Let $\mathfrak{a}$ be a finitely generated ideal in $A$ generated by the set $\{a_1,a_2,...,a_n\}$. We note the following properties expressed by the Boolean conditions above show us that the element
  $$ m = \sum_{i=1}^n a_i $$ Note that we can express any of the $a_i$ by multiplying $m$ with $a_1...a_{i-1}a_{i+1}...a_n$

 \end{enumerate}

\end{proof}

\subsection{Problem 14}
In a ring $A$, let $\Sigma$ be the set of ideals where all of its elements are zero divisors. Show that $\Sigma$ has a maximal element and that this maximal element is a prime ideal. Hence, the set of zero-divisors is a union of prime ideals.

\begin{proof}
 Order $\Sigma$ by set inclusion. For any chain of set inclusions $\mathfrak{a_1} \subseteq \mathfrak{a_2} \subseteq ... \subseteq \mathfrak{a_n}$, the union 
$$ \mathfrak{a}_m = \bigcup \mathfrak{a_i} $$ will be the maximal element in this finite chain as every element will be a zero-divisor in $R$. By Zorn's Lemma, we conclude that $\Sigma$ has a maximal element denoted as $\Sigma_m$. Suppose that $x,y\in R$ are elements such that $xy \in \Sigma_m$.
Then $xyb = 0$ for some $b \neq 0$. However, $x(yb) = 0$ and $x,yb \neq 0$, making $x$ a zero-divisor. So $x \in \Sigma_m$. By symmetry, we can also arrive that $y \in \Sigma_m$. If both $x,y \in \Sigma_m$, this would contradict the maximality of $\Sigma_m$.
\end{proof}

Compare this to the definition using the annhilator of $x \in R$. If $D$ is the set of the zero-divisors in $R$
$$D = \bigcup_{x \in R} r(\text{Ann}(x))$$

\section{The Prime Spectrum of a Ring}

\subsection{Problem 15}
Let $A$ be a ring and let $X$ be the set of all prime ideals of $A$. For each $E \subseteq A$, denote $V(E) \subseteq X$ be the set of prime ideals containing $E$.
Prove that 
\begin{enumerate}
 \item If $\mathfrak{a}$ is the ideal generated $E$, then $V(E)= V(\mathfrak{a}) = V(r(\mathfrak{a}))$
 \begin{proof}
  If a prime ideal contains subset $E$, then it must contain the smallest ideal generated by $E$. Recall that the ideal generated by $E$ can be defined as 
  $$ \bigcap_{E \subseteq I \text{ ideal}} I$$ Hence $V(\mathfrak{a}) \subseteq  V(E)$. Since we are considering sets of prime ideals, $V(r(\mathfrak{a})) \subseteq  V(\mathfrak{a}) $ as $x^n \in \mathfrak{a} \subseteq \mathfrak{p}$ for some $\mathfrak{p}$ prime and $n > 0$ implies that $x \in \mathfrak{p}$. Since $E \subset \mathfrak{a}$, it is clear that $V(E) \subseteq V(r(\mathfrak{a}))$ which yields the equality.
 \end{proof}
\item $V(0) = X$, $V(1) = \emptyset$
\begin{proof}
 $V(0) = X$ is clear from the definition of an ideal. $V(1) = \emptyset$ no prime ideal can contain the entire ring $R$ by definition.
\end{proof}

\item If $(E_i)_{i \in I}$ is any family of subsets (say countable for now), then
$$ V( \bigcup_{i \in I} E_i) = \bigcap_{i \in I} V(E_i) $$
\begin{proof}
 The inclusion $$ \bigcap_{i \in I} V(E_i) \subseteq  V( \bigcup_{i \in I} E_i) $$
 follows from the observation that for any $E_i$, $$ V(E_i) \subseteq V(\bigcup_{i \in I} E_i) $$ for any prime ideal that contains the collective union of $\{E_i\}$ must contain each individual set.
 On the other hand, if a prime ideal $\mathfrak{p}$ contains the collective union $\{E_i\}$ then $$\mathfrak{p} \in V(E_i)$$ for all $i \in I$. Thus, $$ \mathfrak{p} \in \bigcap_{i \in I} V(E_i) $$ and we have the converse inclusion.
\end{proof}

\item Let $\mathfrak{a},\mathfrak{b}$ be any ideals in $R$.

$V(\mathfrak{a} \cap \mathfrak{b}) \subseteq V(\mathfrak{a}\mathfrak{b})$ since $\mathfrak{a} \cap \mathfrak{b} \subseteq \mathfrak{ab}$. The inclusion $V(\mathfrak{a}  \mathfrak{b}) \subseteq V(\mathfrak{a}) \cup V(\mathfrak{b})$ is clear as $V(\mathfrak{a}  \mathfrak{b}) \subseteq V(\mathfrak{a}) $.

Finally, $V(\mathfrak{a}) \cup V(\mathfrak{b}) \subseteq V(\mathfrak{a} \cap \mathfrak{b})$ as $V(\mathfrak{a}) \subseteq V(\mathfrak{a} \cap  \mathfrak{b})$ a la Problem 3.

We see that this defines a topology on the set of prime ideals as the sets $V(E)$ are defined as closed sets in the topology as part 4 shows us that they are closed under finite unions. This topology is called the Zariski topology and the topological space $X$ is the spectrum of the ring $A$ denoted as Spec($A$)
\end{enumerate}

Let us consider Spec($\mathbb{Z}$). The prime ideals are exactly the principal ideals $(p)$ for $p$ prime. Consider any finite set $E \subset \mathbb{Z}$. We see from the axioms of the Zariski topology on Spec($\mathbb{Z}$) that 
$$ V(E) = V(\bigcup_{n \in E} \{ n\}) = \bigcap_{n \in E} V(n) $$
Let $n = p_1^{r_1}...p_r^{r_n}$ be the prime decomposition of $n$
then $(p_i) \in V(n)$ for $1 \leq i \leq r$. Thus, the intersection will contain all the prime divisors  all the elements in $E$. 

\end{document} 
