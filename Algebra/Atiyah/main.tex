\documentclass[12pt]{article}
\setlength{\oddsidemargin}{0in}
\setlength{\evensidemargin}{0in}
\setlength{\textwidth}{6.5in}
\setlength{\parindent}{0in}
\setlength{\parskip}{\baselineskip}

\usepackage{amsmath,amsfonts,amssymb,amsthm}
\usepackage{subfiles}

\title{Problems in Atiyah's Commutative Algebra}
\author{Edward Kim}

\begin{document}
\maketitle
\clearpage

\section{Extensions and Contractions}

Let $A \rightarrow B$ be a ring homomorphism. For an ideal $\mathfrak{a}$,  $f(\mathfrak{a})$ is not an ideal in general.
Consider the embedding $\phi: \mathbb{Z} \rightarrow \mathbb{Q}$, and take $f((m))$ for any $m \in \mathbb{Z}$. In general, for any
$q \in \mathbb{Q}$, $qf(m) \not\in (f((m)))$. To complete the image, we take the
extension $\mathfrak{a}^e$ to be the ideal generated by $f(\mathfrak{a})$ in $B$. We define the contraction of an ideal $\mathfrak{b}^c$ in $B$ to be
$f^{-1}(B)$ which is always an ideal in $A$.

\subsection{Properties of Extensions and Contractions}

The following operations are closed under extensions and contractions. Let $\mathfrak{r(a)}$ denote the radical of ideal $\mathfrak{a}$. 
\begin{enumerate}
 \item $(\mathfrak{a}_1 + \mathfrak{a}_2)^e = Bf(\mathfrak{a}_1 + \mathfrak{a}_2) = Bf(\mathfrak{a}_1) + Bf(\mathfrak{a}_2) = \mathfrak{a}_1^e + \mathfrak{a}_2^e$
 \item $(\mathfrak{a}_1 \cap \mathfrak{a}_2)^e \subseteq \mathfrak{a}_1^e \cap \mathfrak{a}_2^e$. Easily follows since if $x \in (\mathfrak{a}_1 \cap \mathfrak{a}_2)^e$, then
 $x \in \mathfrak{a}_1^e$ and  $x \in \mathfrak{a}_2)^e$. Howeverm, the converse is generally not true. $Let \phi: \mathbb{Z} \rightarrow \mathbb{Q}$ be the natural embedding and let
 $\mathfrak{a} = (p_1)$ and $\mathfrak{b} = (p_2)$ for two primes $p_1,p_2$. Then $\phi(\mathfrak{a}) \cap \phi(\mathfrak{b})$ (How about the example with the map $\mathbb{Z} \rightarrow \mathbb{Z}[i]$ with
 $\mathfrak{a} = (2)$ and $\mathfrak{b} = (5)?)$
 \item $(\mathfrak{a}_1\mathfrak{a}_2)^e = Bf(\mathfrak{a}\mathfrak{b}) = Bf(\mathfrak{a})f(\mathfrak{b}) =Bf(\mathfrak{a})Bf(\mathfrak{b}) = \mathfrak{a}^e\mathfrak{b}^e$
 \item $(\mathfrak{a} : \mathfrak{b})^e \subseteq (\mathfrak{a}^e : \mathfrak{b}^e)$ ( Let $y \in (\mathfrak{a} : \mathfrak{b})$, then 
 $Bf(y)f(\mathfrak{b}) \subseteq f(\mathfrak{a}))$
 \item $\mathfrak{r(a)}^e \subseteq \mathfrak{r(a^e)}$. If $y \in r(\mathfrak{a})^e$, then $y = bx$ where $x \in \mathfrak{r(a)}$. Thus, $ y \in Bf(\mathfrak{r(a)})$ and $y \in \mathfrak{r(a)}^e$
 \end{enumerate}

 \section{Exercises}
 
 \subsection{Problem 1}
 \newpage
 Question: If $x$ is a nilpotent element in commutative ring $A$, show that $1 + x$ is a unit. Deduce that the sum of a nilpotent element and a unit is a unit.
 \begin{proof}
 Since $x \in \mathfrak{R}$ where $\mathfrak{R}$ is the nilradical, there exists $n > 0$ such that $x^n = 0$.
 Multiplying by $x^{n-1}$ yields $x^{n-1}(1 + x) = x^{n-1} + 0 = x^{n-1}$. Thus, $1 + x$ must be a unit. We can generalize this result as follows:
 Let $u$ be a unit in the ring $A$ and $x$ as defined above. Performing a similar operation as above, we receive
  $$ x^{n-1}(u + x) = ux^{n-1} $$. As $u$ is a unit, let $y \in R$ be chosen such that $yu = 1$. Multiplying $y$ on both sides gives us: $$ x^{n-1}(1 + yx) = x^{n-1} $$. As $yx \in \mathfrak{R}$, we see that the addition of a unit with a nilpotent element gives us a unit in $R$.
 \end{proof}

Problem 1 relates to the following results from Problem 10:

\subsection{Problem 10}

Question: Let $A$ be a ring and $\mathfrak{R}$ be its nilradical. Show that the following are equivalent:
\begin{enumerate}
 \item Show that $A$ has exactly one prime ideal.
 \item Every element of $A$ is either a unit or nilpotent.
 \item A / $\mathfrak{R}$ is a field.
\end{enumerate}

\begin{proof}
 For the implication $2 \implies 3$. under the assumption of 2, the non-trivial cosets of $A / \mathfrak{R}$ will all contain units.
 Thus, for any $\bar{u} \in A / \mathfrak{R}$, the multiplication of coset of $\bar{u^{-1}}$ will yield $\bar{1}$. Thus, $A / \mathfrak{R}$ is a field. \par
 For $3 \implies 1$, If $A / \mathfrak{R}$, then $\mathfrak{R}$ must be maximal. Thus, $\mathfrak{R}$ is prime. However as $\mathfrak{R}$ is the intersection of all prime ideals, it follows that it must be the only prime ideal.
 For $1 \implies 2$, by proposition 1.9, we know that any $x \in A - \mathfrak{R}$ must be a unit as if there is only one prime ideal, it must exactly be a maximal ideal. Thus, $A$ is only composed of units and nilpotent elements.
\end{proof}


Remark: This ring seems to be partitioned into two classes: one with elements that are inextricably linked to the muliplicative identity 1 and the other consisting of elements eventually return to 0.
 
\subsection{Problem 2}

Let $A$ be a commutative ring and $A[x]$ its corresponding polynomial ring with an indeterminate $x$ with coefficients in $A$.
\begin{enumerate}
 \item 
\end{enumerate}


\end{document} 
