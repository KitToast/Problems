\documentclass[Lang.tex]{subfiles}

\begin{document}

\section*{Lang Chapter 1 (Groups)}
\subsection*{Problem 6}

Question: Prove that the group of inner automorphisms of a group $G$ are normal in $Aut(G)$. \par
\begin{proof}
Let $\mathcal{L} \in Aut(G)$ and $Inn(G)$ be the subgroup of inner automorphisms of $G$. It follows from our definitions that for any $y \in G$,  $$(\mathcal{L} \circ Inn(G) \circ \mathcal{L}^-1)(y) = (\mathcal{L} \circ Inn(G)) (\mathcal{L}^-1(y)) = \mathcal{L}(x \mathcal{L}^{-1}(y) x^{-1}) = \mathcal{L}(x) y \mathcal{L}^{-1}(x) \in Inn(G)$$

Since $x \in G$ was arbitrary, we see that $\mathcal{L} \circ Inn(G) \circ \mathcal{L}^-1 \subseteq Inn(G)$. Hence, $Inn(G) \trianglelefteq Aut(G)$. \end{proof}

\subsection*{Problem 7}

Question: Let $G$ be a group such that $Aut(G)$ is cyclic. Prove that $G$ is abelian.

\begin{proof}
By Proposition 4.2 (Lang 42), the group of inner of automorphisms $Inn(G)$ is also cyclic. Let $G \rightarrow Inn(G)$ be the surjective homomorphism $x \mapsto c_x$. The kernel of this map is the center of $Z(G)$ of the group. Hence, $G/Z(G) \cong Inn(G)$. It suffices to prove that if $G/Z(G)$ is abelian, then $G$ must be abelian. We can simply take this in two cases. If we have $a,b \in G$, let's take the case where they both occupy the same coset. Thus, $ab^{-1} \in Z(G)$. We can use this for the following equivalences:
 $$ ab = a(b^{-1}b)b = (ab^{-1})bb = b(ab^{-1})b = ba$$
 Hence, any two elements in the same coset must commute. As the factor group is cyclic, let $x$ be a generator for $G/Z(G)$. Then given that $a,b \in G$ live in different cosets, $a,b$ must occupy a coset with some power of $x$. Since the powers of $x$ trivially commute and $a,b$ commute with the generators, it follows that $a,b$ must commute and the proposition follows.
\end{proof}

\subsection*{Problem 9}

\begin{enumerate}
\item 
Question: Let $G$ be a group and $H$ a subgroup of finite index. Show that there exists a normal subgroup $N$ of $G$ contained in $H$ and also of finite index.

\begin{proof}
Let $(G : H) = n$. We can construct a homomorphism $\phi: G \rightarrow S_n$ by conjugating the coset representatives labeled up to $n$ for all $g \in G$. It follows that $Ker(\phi) \subseteq H$. Let $N = Ker(\phi)$. Then by our isomorphism theorems, $G/N$ is isomorphic to it's image which is finite and the proposition follows.   
\end{proof}

\item
Question: Let $G$ be a group and let $H_1,H_2$ be subgroups of finite index. Prove that $H_1 \cap H_2$ has finite index.

\begin{proof}
Consider the cosets of $G/H_1$. By assumption, the number of cosets $(G:H_1)$ is finite. Pick any coset $C$ and consider how the elements partition into the cosets of $(G/(H_1 \cap H_2)$ by cases. Let $a,b \in C$ be any two elements of our coset. If $ab^{-1} \in H_2$, then it aligns with a coset of $H_2$. Otherwise, the two elements split into separate cosets in $G/(H_1 \cap H_2)$, namely a coset where $ab^{-1} \in H_2$ as well. However, the number of cosets it could move into is bounded by $(G:H_2)$. Thus, the number of unique cosets that could be generated by this method is bounded by $(G:H_1)(G:H_2)$ which is finite. Hence, $H_1 \cap H_2$ is of finite index. \end{proof}
\end{enumerate}

\subsection{Problem 12 (Semidirect Product)}

\begin{enumerate}
\item $x \mapsto \gamma_x$ induces a homomorphism $f: H \rightarrow Aut(N)$ since $$f(yz) = x(yz)x^{-1} = (xyx^{-1})(xzx^{-1}) = f(x)f(y)$$
\item 
We first note that the kernel of the map $\phi: H \times N \rightarrow HN$ is trivial. Let $f(e,n_1) = e$, then
$en = e$ and $n = e$. A symmetric argument can be done on the left. Since $H \cap N = \{e\}$, it follows that the kernel is trivial. Surjection follows immediately from the definition of the map. Hence, $\phi$ is a bijective map. To show that this map is an isomorphism of groups, it suffices to prove that $\phi$ is a homomorphism if and only if $f$ is the trivial map. Starting with the converse, if $f$ is the trivial map, then $hnh^{-1} = n$ for any $n \in N, h \in H$. Hence,
\begin{align*}
     & \phi(h_1h_2,n_1n_2) = (h_1h_2)(n_1n_2) = (h_1h_2)(n_1 (h_2^{-1}h_2) n_2) = \\
     &h_1 (h_2n_1h_2^{-1})h_2 n_2 = (h_1n_1)(h_2n_2) - \phi(h_1,n_1)\phi(h_2,n_2)
\end{align*}
Now if $\phi$ is a homomorphism, then by our observation above,
$$  (h_1h_2)(n_1n_2) = (h_1n_1)(h_2n_2)  $$
Taking the proper left and right inverses yield the following:
$$ h_2 n_1 h_2^{-1} = n_1 $$
By taking $h_2,n_1$ to be any element in $H,N$ respectively, we conclude that $f$ is indeed trivial.
\item
Let $N,H$ be subgroups. Define $\psi: N \rightarrow Aut(H)$ to be our above homomorphism. We will construct the semidirect product as follows: Let $x \in N, h \in H$ and construct the set made of elements of the form $(x,h)$. Define the composition law as follows:
$$ (x_1,h_1)(x_2,h_2) = (x_1\psi(h_1) x_2, h_1h_2)$$

We will first show that this indeed follows the group laws. Associativity easily follows from the underlying property of groups $N,H$. The identity exists from $e_N,e_H,\psi(e_N) = id_H$. The inverse also exists since $$(x,h)(\psi^{-1}(h)x^{-1},h^{-1}) = (e_N,e_H)$$

Hence, the set is a group. We will now show that this yields a semidirect product on $N$ and $H$ by associating $N$ with $(n,1)$ and $H$ with $(1,h)$. 

We immediately note that $N \cap H = \{ e\}$. Thus, it suffices to prove that $NH$ is isomorphic to our group through a map $\theta: NH \rightarrow G, nh \mapsto (n,h)$. First, we show that $\theta$ is indeed a homomorphism. 


\end{enumerate}

Note: We can extend this definition for $H$ be the normalizer of any subgroup $N$. Then the map $f$ would be the homomorphism from $Norm(N) \rightarrow Aut(N)$. \par

Note: We note why the map $f$ above is so significant. What's stopping the map $H \times N \rightarrow HN$ is precisely the conjugation issue $(hnh^{-1} = n)$ for all $n \in N, h \in H$.

\subsection*{Problem 17}

Question: Let $X,Y$ be finite sets and let $C \subseteq X \times Y$. For $x \in X$, let $\phi(x) = | \{y \in Y \vert (x,y) \in C\}|$. Verify that $|C| = \sum_{x \in X} \phi(x)$

\begin{proof}
	Let $|X| = n_1$ and $|Y| = n_2$. 
\end{proof}

\subsection*{Problem 18}

\begin{proof}
	We use Problem 17 by taking our set to be $S \times T$ and our subset $C$ to be the entirety of $S \times T$. The result follows from the finite sum shown above.
\end{proof}

\subsection*{Problem 19}

\begin{enumerate}
	\item 
	\begin{proof}
		This follows by weighting the cardinality of orbit $G_s$ for each element. 	
	\end{proof}
	\item 
	\begin{proof}
	By Lagrange's Theorem, we know that $|G:G_s| = |G|/|G_s|$ where $G_s$ is the stabilizer of $s \in S$. By the part above: 
	$$\sum_{t \in Gs} \frac{1}{|Gt|} = \frac{1}{|G|}\sum_{t \in Gs} |G_t| = 1 $$
	Hence, we can add this sum over all coset representatives $S_r$:
	$$ \frac{1}{|G|} \sum_{s_r \in \mathcal{O}} \sum_{t \in Gs_r} |G_t| = |\{\mathcal{O}_i\}|$$
	where $\{\mathcal{O}_i\}$ is the set of orbits of $G$ in $S$. Finally, we note that the following sums are equivalent:
	$$\sum_{s \in Gs} |Gs| = \sum_{g \in G} f(x)$$ where $f(x)$ is the number of fixed points $x \in G$ exhibits. Substituting the sum yields our desired result:
	$$ \frac{1}{|G|} \sum_{g \in G} f(x) = |\{\mathcal{O}_i\}|$$	   
	\end{proof}	
\end{enumerate}

\end{document}