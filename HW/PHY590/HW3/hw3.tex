\documentclass[12pt]{article}%
\usepackage{amsfonts}
\usepackage{amsmath}
\usepackage[a4paper, top=2.5cm, bottom=2.5cm, left=2.2cm, right=2.2cm]%
{geometry}
\usepackage{times}
\usepackage{amsmath}
\usepackage{amssymb}
\usepackage{physics}
\newenvironment{proof}[1][Proof]{\textbf{#1.} }{\ \rule{0.5em}{0.5em}}

\begin{document}

\title{PHYS 590 HW 3}
\author{Edward Kim}
\date{\today}
\maketitle

\section*{Problem 1}
\begin{enumerate}
  \item
  The equality follows from the definition of mutual information along with the assumption that our $\psi_{ABC}$ is pure. First, observe the following string of equalities:

  \begin{gather*}
    I(A;B)_{\psi} + I(A;C)_{\psi} = \\
    H(A)_{\psi} + H(B)_{\psi} - H(AB)_{\psi} + H(A)_{\psi} + H(C)_{\psi} - H(AC)_{\psi}
  \end{gather*}
  Since $\psi_{ABC}$ is a pure state:
  \begin{gather*}
    H(AB)_{\psi} = H(C)_{\psi} \\
    H(AC)_{\psi} = H(B)_{\psi} \\
    H(BC)_{\psi} = H(A)_{\psi}
  \end{gather*}

  Substituting these equalities above gives us that:

  \begin{gather*}
    H(A)_{\psi} + H(BC)_{\psi}
  \end{gather*}

  Finally, since $H(ABC)_{\psi} = 0$ this yields the desired equality:
  $$  I(A;B)_{\psi} + I(A;C)_{\psi} = \boxed{I(A;BC)_{\psi}} $$

\item
As previously, we decompose the expression through definitions:
\begin{gather*}
    I(A;B\vert C)_{\psi} = H(A\vert C)_{\psi} + H(B \vert C)_{\psi} + H(AB \vert C)_{\psi} = \\
    [ H(AC)_{\psi} - H(C_{\psi})] + [H(BC)_{\psi} - H(C)_{\psi} + [H(ABC)_{\psi} - H(C_{\psi})]
\end{gather*}
By purity of $\psi_{ABCD}$:

\begin{gather*}
  H(AC)_{\psi} = H(BD)_{\psi} \\
  H(AD)_{\psi} = H(BC)_{\psi} \\
  H(ABC)_{\psi} = H(D)_{\psi} \\
  H(C)_{\psi} = H(ABD)_{\psi}
\end{gather*}

Substituting these equalities above, we yield that:
\begin{gather*}
  H(BD)_{\psi} + H(AD)_{\psi} - H(ABD)_{\psi} - H(D)_{\psi} = \\
  [H(BD)_{\psi} - H(D)_{\psi}] + [H(AD)_{\psi} - H(D)_{\psi}] - [H(ABD)_{\psi} - H(D)_{\psi}] = \\
  H(A \vert D)_{\psi} + H(B \vert D)_{\psi} - H(AB \vert D)_{\psi} = \\
  \boxed{I(A;B\vert D)_{\psi}}
\end{gather*}

\section*{Problem 2}
The corresponding density matrix for $\rho_{AB}(p)$ is expressed as:
$$
\rho_{AB}(p) = \left[\begin{matrix}
  \frac{p}{4} & 0 & 0 & 0 \\
  0 & \frac{1}{2} - \frac{p}{4} & \frac{1-p}{2}  & 0 \\
  0 & \frac{1-p}{2} & \frac{1}{2} - \frac{p}{4} & 0\\
  0 & 0 & 0 & \frac{p}{4}
\end{matrix} \right]$$
Calculating the spectra directly shows that:
$$\lambda_1 = \frac{p}{4}, \; \lambda_2 = \frac{p}{4}, \; \lambda_3 = 1 - \frac{3p}{4}, \; \lambda_4 = 1 - \frac{3p}{4} $$
 \end{enumerate}
 Thus, the joint quantum entropy is given by:
 $$ H(AB)_{\rho(p)} =  - 2(\frac{p}{4})\log_2{(\frac{p}{4})} - 2 (1 - \frac{3p}{4})\log_2{(1- \frac{3p}{4})}$$
 Finally, we calculate $\rho_A(p) = Tr_B(\rho_{AB}(p))$:
 \begin{gather*}
   Tr_B(\rho_{AB}(p)) =
     \left[
        \begin{matrix}
          \frac{p}{4} & 0 \\
          0 & \frac{1}{2} - \frac{p}{4}
        \end{matrix}
     \right] +
     \left[
        \begin{matrix}
          \frac{1}{2} - \frac{p}{4} &  \\
          0 & \frac{p}{4}
        \end{matrix}
     \right] =
     \left[
        \begin{matrix}
          \frac{1}{2} &  \\
          0 & \frac{1}{2}
        \end{matrix}
     \right]
 \end{gather*}
 Thus, this shows us that:
 $$ H(A)_{\rho(p)} = 1 $$
 Finally. combining the two results above gives the desired expression:
 $$ H(B \vert A)_{\rho(p)} =  - 2(\frac{p}{4})\log_2{(\frac{p}{4})} - 2 (1 - \frac{3p}{4})\log_2{(1- \frac{3p}{4})} - 1$$
 Directly solving for the threshold value allows us to observe that
 $$ p_{th} = \frac{3}{10} $$
 In other words, as long as $p \geq \frac{3}{10}$, the conditional quantum entropy $H(B \vert A)_{\rho(p)}$ will be non-negative. Furthermore, when $p =1$:
 \begin{gather*}
   - 2(\frac{1}{4})\log_2{(\frac{1}{4})} - 2 (\frac{1}{4})\log_2{( \frac{1}{4})} - 1 = 2 - 1 = 1
 \end{gather*}
 showing us that when $p=1$, we have positive conditional entropy.

\section*{Problem 3}
First, recall that:
\begin{equation*}
  I(A_1A_2; B_1B_2)_{\omega} = H(A_1A_2)_{\omega} + H(B_1B_2)_{\omega} - H(A_1A_2B_1B_2)_{\omega}
\end{equation*}
By subadditivity and the fact that $\omega_{A_1A_2B_1B_2} = \rho_{A_1B_1}\otimes \rho_{A_2B_2}$:
$$H(A_1A_2B_1B_2)_{\omega} = H(A_1B_1)_{\omega} + H(A_2B_2)_{\omega} =
H(A_1B_1)_{\rho} + H(A_2B_2)_{\sigma}
$$
However, pairs of states $A_1,A_2$ and $B_1,B_2$ are uncorrelated in the state $\omega_{ABCD}$, allowing us to see that:
\begin{gather*}
  H(A_1A_2)_{\omega} = H(A_1)_{\rho} + H(A_2)_{\sigma} \\
  H(B_1B_2)_{\omega} = H(B_1)_{\rho} + H(B_2)_{\sigma}
\end{gather*}
Combining these inequalites gives us the desired result:

\begin{gather*}
  \begin{split}
    H(A_1A_2)_{\omega} + H(B_1B_2)_{\omega} - H(A_1A_2B_1B_2)_{\omega} = & H(A_1)_{\rho} + H(A_2)_{\sigma} \\+ & H(B_1)_{\rho} + H(B_2)_{\sigma} \\ -
    & H(A_1B_1)_{\rho} - H(A_2B_2)_{\sigma}
  \end{split}\\
  = [H(A_1)_{\rho} + H(B_1)_{\rho} - H(A_1B_1)_{\rho}] - [H(A_2)_{\rho} + H(B_2)_{\rho} - H(A_2B_2)_{\rho}] \\
  = \boxed{I(A_1; B_1) + I(A_2; B_2)}
\end{gather*}

\section*{Problem 4}
\begin{enumerate}
  \item
  \begin{enumerate}
    \item
      For $\rho_{ABC} = \frac{1}{2}(\ket{000}_{ABC} + \ket{111}_{ABC})$
      \begin{gather*}
        I(A;C)_{\rho} = H(A)_{\rho} + H(C)_{\rho} - H(AC)_{\rho} = 2 - 1 = \boxed{1} \\
        I(A; C\vert B) = H(A \vert B)_{\rho} + H(C \vert B)_{\rho} - H(AC \vert B)_{\rho} \\
        = [H(AB)_{\rho} - H(B)_{\rho}] + [H(BC)_{\rho} - H(B)_{\rho}] - [H(ABC)_{\rho} - H(B)_{\rho}] = 0 + 0 + 0 = \boxed{0}
      \end{gather*}
      \item
      For $\rho_{ABC} = \frac{1}{2}(\ket{000}_{ABC} + \ket{111}_{ABC})(\bra{000}_{ABC} + \bra{111}_{ABC})$:
      \begin{gather*}
        I(A;C)_{\rho} = 2 - 1 = \boxed{1} \\
        I(A; C\vert B) = 0 + 0 + 1 = \boxed{1}
      \end{gather*}
      \item
      For $\rho_{ABC} = \frac{1}{4}\sum_{x,y \in \{0,1\}} \ket{x}\bra{x} \otimes \ket{y}\bra{y} \otimes \ket{x \oplus y}\bra{x \oplus y} $,
      \begin{gather*}
        I(A;C) = H(A)_{\rho} + H(C)_{\rho} - H(AC)_{\rho} = 2 - 2 = \boxed{0} \\
        I(A; C\vert B) = \\ [H(AB)_{\rho} - H(B)_{\rho}] + [H(BC)_{\rho} - H(B)_{\rho}] - [H(ABC)_{\rho} - H(B)_{\rho}] \\
        = [2-1] + [2-1] - [2-1] =\boxed{1}
      \end{gather*}
  \end{enumerate}
  Comparing $I(A;C)$ with $I(A;C\vert B)$, we see that the first three obeys the expression $I(A;C) \geq I(A;C\vert B)$. However, the last one does not.
  (I'm a bit confused by the last result to be honest).

  \item The state $\omega_{BAC}$ is a generalization of the classical-quantum state except here Alice can prepare a bipartite state by picking local unitary operators to act on each respective state. If we let $\zeta_{AC}^{(k)} = \rho_A^{(k)} \otimes \sigma_C^{(k)}$, we can rephrase our state as
  $$\omega_{BAC} = \sum_{k} p_k \ket{k}\bra{k} \otimes \zeta_{AC}^{(k)} $$\
  Now we know that
  $$ I(A;C\vert B)_{\omega} = \sum_k p_k I(A;C)_{\zeta^{(k)}} $$ This simply follows from the definition of classical conditional entropy. However, since $\zeta^{(k)}_{AB}$ is a product state for all $k$:
  $$ I(A;C)_{\zeta^{(k)}} = 0 $$
  Thus, we have that $I(A;C\vert B) = 0$.
  \item
  We will utilize the chain rule for quantum mutual informtion:
  \begin{equation*}
    I(A;BC)_{\rho} = I(A;B)_{\rho} + I(A;C\vert B)_\rho
  \end{equation*}
  From definitions:
  \begin{gather*}
    I(A;C\vert B_1B_2)_\omega = I(A;CB_1\vert B_2)_\omega - I(A;B_1 \vert B_2)_\omega \\
    = I(CB_1,AB_2)_\omega - I(CB_1,B_2)_\omega - I(A;B_1\vert B_2)_\omega \\
    = I(CB_1,AB_2)_\omega - I(CB_1,B_2)_\omega - I(AB_2;B_1) + I(B_1;B_2)_\omega \\
    = I(CB_1,AB_2)_\omega - I(C,B_2)_{\rho_{B_2C}} - I(A;B_1)_{\rho_{AB_1}}
  \end{gather*}
  The last equality follows from the observation that $B_1,B_2$ are uncorrelated in state $\omega_{AB_1B_2C}$ (tracing over $A$ and $C$ will yield a the product state $\rho_{B_1} \otimes \rho_{B_2}$). For similar reasons, $B_1,C$, $B_2,A$ are also uncorrelated in $\omega_{AB_1B_2C}$.
  Now by additivity of quantum mutual information:
  \begin{equation*}
    I(AB_2,CB_1)_{\omega} = I(A;B_1)_{\rho_{AB_1}} + I(B_2;C)_{\rho_{B_2C}}
  \end{equation*}
  the final equality vanishes, so $I(A;C \vert B_1B_2) = 0$.
\end{enumerate}

\end{document}
