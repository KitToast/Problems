\documentclass[12pt]{article}%
\usepackage{amsfonts}
\usepackage{amsmath}
\usepackage[a4paper, top=2.5cm, bottom=2.5cm, left=2.2cm, right=2.2cm]%
{geometry}
\usepackage{times}
\usepackage{amsmath}
\usepackage{amssymb}
\newenvironment{proof}[1][Proof]{\textbf{#1.} }{\ \rule{0.5em}{0.5em}}
\setlength{\parindent}{0em}


\begin{document}

\title{Math 774 Homework 6}
\author{Edward Kim}
\date{\today}
\maketitle

\section*{Problem 1 (Humphreys 13.4)}
\begin{proof}
We begin with the case for $A_{\ell}$. From Table 1, we know the following linear relations between $\lambda_i$ and $\alpha_j$ for $1 \leq i,j \leq \ell$ :
$$\lambda_i = \frac{1}{\ell + 1}[(\ell - i +1)\alpha_1 + 2(\ell -i + 1)\alpha_2 + ... + (i-1)(\ell -i + 1)\alpha_{i-1} + i(\ell -i + 1)\alpha_i + i(\ell -i + 1)\alpha_{i+1} + ... + i \alpha_{\ell}]$$
Note that since $i < \ell + 1$, $c\lambda_i \not\in \Lambda_r$ for $1 \leq c \leq \ell$ and $(\ell + 1)\lambda_i \in \Lambda_r$ where $\Lambda_r$ is the subgroup of $\Lambda$ generated by elements of $\Phi$. From these observations and the fact that the $\lambda_i$ span $\Lambda$, it follows that $\Lambda / \Lambda_r$ is cyclic of order $\ell + 1$.

For $D_{\ell}$, we once again state the relations between the $\lambda_i$,$\alpha_i$:
\begin{gather*}
  \lambda_i = \alpha_1 + 2\alpha_2 + ... + (i-1)\alpha_{i-1} + i(\alpha_i + ... + \alpha_{\ell -2}) + \frac{1}{2}i(\alpha_{\ell-1} + \alpha_{\ell}), \quad (i < \ell -1) \\
  \lambda_{\ell - 1} = \frac{1}{2}[\alpha_1 + 2\alpha_2 + ... + (\ell -2)\alpha_{\ell -2} + \frac{1}{2}\ell\alpha_{\ell-1} + \frac{1}{2}(\ell-2)\alpha_{\ell}] \\
  \lambda_{\ell} = \frac{1}{2}[\alpha_1 + 2\alpha_2 + ... + (\ell -2)\alpha_{\ell -2} + \frac{1}{2}(\ell-2)\alpha_{\ell-1} + \frac{1}{2}\ell\alpha_{\ell}]
\end{gather*}

First if $\ell$ is odd, then $\ell - 2$ must also be odd. We then see that we have terms $\alpha_{\ell-1},\alpha_{\ell}$ with coefficients $\frac{1}{4}$ in $\lambda_{\ell-1},\lambda_{\ell}$. Thus, the formulae above, we see that the $[\lambda_i] \in \Lambda / \Lambda_r$ for $i < \ell - 1$ have order 2 and both $[\lambda_{\ell - 1}],[\lambda_{\ell}] \in \Lambda / \Lambda_r$ have order 4. \newline

By generally considering any $\lambda \in \Lambda$ as $\lambda = \sum_{i=1}^{\ell} c_i \lambda_i, c_i \in \mathbb{Z}$, our remarks above show that $[\lambda] = \sum_{i=1}^{\ell} c_i'[\lambda_i]$ where $0 \leq c_i' \leq 3$. \newline

For $\ell$ even, $\ell - 2$ will also be even. The terms $\alpha_{\ell-1},\alpha_{\ell}$ in $\lambda_{\ell - 1},\lambda_{\ell}$ will have coefficients $\frac{1}{2}$. Thus, once again $[\lambda_i] \in \Lambda / \Lambda_r$ for $i < \ell - 1$ have order 2, but $[\lambda_{\ell - 1}],[\lambda_{\ell}] \in \Lambda / \Lambda_r$ will have order 2 as well.
\end{proof}

\section*{Problem 2 (Humphreys 13.5)}
\begin{proof}
  Let $\Lambda_r \subset \Lambda'$ be a subgroup of $\Lambda$. For $\sigma_\alpha \in \mathcal{W}$ and $\lambda' \in \Lambda'$, $\sigma_{\alpha}(\lambda') = \lambda' - \langle \lambda',\alpha \rangle \alpha$. Since $\langle \lambda',\alpha \rangle \in \mathbb{Z}$ and $\alpha \in \Lambda_r$, $\sigma_{\alpha}(\lambda') \in \Lambda'$, showing that $\Lambda'$ is indeed $\mathcal{W}$-invariant.
\end{proof}

\section*{Problem 3 (Humphreys 20.5)}
\begin{proof}
 From Theorem 20.2(b), we know that all weights $\mu$ of $Z(\lambda)$ are of the form $\lambda - \sum_{i = 1}^{\ell} k_i \alpha_i$ for $k_{i} \in \mathbb{Z}^+$ and the $\alpha_i$ being simple roots. Furthermore, from Theorem 20.2(a), $Z(\lambda)$ is spanned by vectors of the form $y^{i_1}_{\beta_1}y^{i_2}_{\beta_2}...y^{i_m}_{\beta_m}\cdot v^+$ where $\Phi^+ = \{\beta_1,...,\beta_m\}$. We note that the $y_{\beta_i} \in L_{-\beta_i}$ are the negative root vectors in our semisimple Lie algebra $L$. Since $Z(\lambda)$ is the direct sum of all of its weight spaces $Z(\lambda)_{\mu}$, we conclude from Theorem 20.2(a) that $Z(\lambda)_{\mu}$ must be spanned by all vectors of the form $y_{\beta_1}^{k_1}y_{\beta_2}^{k_2}...y_{\beta_m}^{k_m}\cdot v^+$ where $\mu = \sum_{\beta_i \in \Phi^+}k_{\beta_i}\beta_i$. Furthermore, all such vectors must be linearly independent by the following argument:
 Consider the two such vectors in $Z(\lambda)_{\mu}$ such that
  $$\prod_r y^{k_r}_{\beta_r}\cdot v^+ = c\prod_r y^{h_r}_{\beta_r}\cdot v^+, \quad c \in \mathbb{F}$$
  and $\mu = \sum_{\beta_i \in {\Phi^+}} k_i \beta_i = \sum_{\beta_i \in {\Phi^+}} h_i \beta_i$. Applying $\prod_r x^{k_r}_{\beta_r}$ to both sides yields that:
  $$ v^+ = c (\prod_r x^{k_r}_{\beta_r})(\prod_r y^{h_r}_{\beta_r})\cdot v^+ \not\in Z(\lambda)_{\lambda}, \quad c \in \mathbb{F}  $$ This contradicts the definition of maximal vector $v^+$. Hence, from the definition of $\mathcal{P}(\lambda - \mu)$, it follows that $\dim{Z(\lambda_{\mu}) = \mathcal{P}(\lambda - \mu)}$.
\end{proof}

\section*{Problem 4 (Humphreys 20.9)}
\begin{proof}
  Following along the proof stated in the problem, we first prove that if $v \in Z(\lambda)_{\mu}$ such that $\prod_{\alpha \in \Phi^+} x_{\alpha}^{c_{\alpha}}\cdot v = 0$ for all such $c_{\alpha} \in \mathbb{Z}^+$, $v \in Y(\lambda)$ where $Y(\lambda)$ is the unique maximal vector whose existence is proved in Therorem 20.2(d). We construct the submodule $X(\lambda) = Y(\lambda) \oplus (\mathfrak{U}(L)\cdot v)$ by simply closing the action of $\mathfrak{U}(L)$ on $v$. By our assumed property of $v$, there exists no such action of $x \in \mathfrak{U}(L)$ such that $x \cdot v = v^+$. Therefore, $Z(\lambda)_{\lambda} \not\subset X(\lambda)$ and $X(\lambda)$ must be a proper submodule of $Z(\lambda)$. However, this contradicts our assumption that $Y(\lambda)$ is the unique proper submodule of $Z(\lambda)$. \newline

  Conversely, we shall now prove that $Y(\lambda)$ is the span of all such weight vectors $v$ for weights $\mu \neq \lambda$. Suppose $r \in Y(\lambda)$ be a weight vector in $Z(\lambda)_{\gamma}$ such that $r$ is not contained in the span of all such $v$. By Theorem 20.2(b), there must exist some $c_{\alpha}$ such that $\prod_{\alpha \in \Phi^+} x_{\alpha}^{c_{\alpha}}\cdot r = l\cdot v^+$ where $l \neq 0$. However this implies that $v^+ \in Y(\lambda)$, contradicting its definition of being a proper submodule of $Z(\lambda)$. \newline

  This proves that $Y(\lambda)$ is precisely the span of all such weight vectors $v$ with weights $\mu \neq \lambda$.
\end{proof}

\section*{Problem 5 (Humphreys 21.3)}
\begin{proof}
  Suppose that $\lambda \in \Lambda^+$ is dominant. If zero occurs as a weight in $V(\lambda)$, then from Theorem 20.2(b), $0 = \lambda - \sum_{i=1}^{\ell}k_i\alpha_i$ for $\alpha_i \in \Delta$ and $k_i \in \mathbb{Z}^+$. This shows that $\lambda = \sum_{i=1}^{\ell}k_i \alpha_i$ is a sum of roots. Conversely, suppose that $\lambda = \sum \alpha$ is a finite sum of roots. From the construction of $V(\lambda)$, $(\prod_{\alpha}y^{\alpha}) \cdot v^+ \in V_{(\lambda - \sum  \alpha)} = V_0$. Hence, zero will be a weight of $V(\lambda)$.
\end{proof}

\section*{Problem 6 (Humphreys 21.6)}
\begin{proof}

\end{proof}

\section*{Problem 7}
\begin{enumerate}
  \item Suppose $\psi: V \rightarrow V'$ is a map of weight-graded $L$-modules $V,V'$. As a $L$-module homomorphism, 
\end{enumerate}
\end{document}
