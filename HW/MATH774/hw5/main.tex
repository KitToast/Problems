\documentclass[12pt]{article}%
\usepackage{amsfonts}
\usepackage{amsmath}
\usepackage[a4paper, top=2.5cm, bottom=2.5cm, left=2.2cm, right=2.2cm]%
{geometry}
\usepackage{times}
\usepackage{amsmath}
\usepackage{amssymb}
\newenvironment{proof}[1][Proof]{\textbf{#1.} }{\ \rule{0.5em}{0.5em}}
\setlength{\parindent}{0em}


\begin{document}

\title{Math 774 Homework 5}
\author{Edward Kim}
\date{\today}
\maketitle

\section*{Problem 1 (Humphreys 14.2)}
\begin{proof}
  Recall that $\mathfrak{sl}(2,F)$ is generated by the basis elements:
  $$ x = \begin{bmatrix} 0 & 0 \\ 1 & 0 \end{bmatrix} \quad y = \begin{bmatrix} 0 & 1 \\ 0 & 0 \end{bmatrix} \quad h = \begin{bmatrix} 1 & 0 \\ 0 & -1 \end{bmatrix}$$
  We first show that each maximal toral sublgebra must be one dimensional. Proceed by noting that since $[xy] = h, [xh] = 2x, [hy] = -2y$, none of the basis elements commute with each other. Hence, no pair of basis elements are simultaneously diagonizable, so no subset of the basis elements with two or more elements can be a toral subalgebra. However, as $h$ is in diagonal form in respect to the standard basis of $F^2$, a maximal toral subalgebra must contain at least one element. This yields that every maximal toral subalgebra must be one-dimensional. \newline

  Now given two such maximal toral subalgebras $H,H'$, $\dim_F{H^*} = \dim_F{(H')^*} = 1$. If $\Phi,\Phi'$ are the root systems relative to $H,H'$ respectively, then both must be one-dimensional root systems. Define the isomorphism $\psi: \Phi \rightarrow \Phi'$ by sending the single generators to each other. This in turn induces a isomorphism between $H$ and $H'$, say $\pi: H \rightarrow H'$. In particular, $\psi$ induces an isomorphism between the corresponding bases $\Delta,\Delta'$ which must also be one-dimensional by the axioms shown in (10.1). By the isomorphism theorem, there exists an automorphism of $\mathfrak{sl}(2,F)$ extending $\pi$, giving the desired result.
\end{proof}

\section*{Problem 2 (Humphreys 14.5)}
\begin{proof}
   $\mathfrak{sl}(3,F)$ can be decomposed into the following set of root space generators:
   \begin{gather*}
     x_{\alpha} = \begin{bmatrix} 0 & 0 & 0 \\ 1 & 0 & 0 \\ 0 & 0 & 0 \end{bmatrix}, y_{\alpha} = \begin{bmatrix} 0 & 1 & 0 \\ 0 & 0 & 0 \\ 0 & 0 & 0 \end{bmatrix}, h_{\alpha} = \begin{bmatrix} 1 & 0 & 0 \\ 0 & -1 & 0 \\ 0 & 0 & 0 \end{bmatrix} \\
     x_{\beta} = \begin{bmatrix} 0 & 0 & 0 \\ 0 & 0 & 0 \\ 0 & 1 & 0 \end{bmatrix}, y_{\alpha} = \begin{bmatrix} 0 & 0 & 0 \\ 0 & 0 & 1 \\ 0 & 0 & 0 \end{bmatrix}, h_{\beta} = \begin{bmatrix} 0 & 0 & 0 \\ 0 & 1 & 0 \\ 0 & 0 & -1 \end{bmatrix} \\
     x_{\alpha + \beta} =  \begin{bmatrix} 0 & 0 & 0 \\ 0 & 0 & 0 \\ 1 & 0 & 0 \end{bmatrix}, y_{\alpha + \beta} = \begin{bmatrix} 0 & 0 & 1 \\ 0 & 0 & 0 \\ 0 & 0 & 0 \end{bmatrix}, h_{\alpha + \beta} = h_{\alpha} + h_{\beta} = \begin{bmatrix} 1 & 0 & 0 \\ 0 & 0 & 0 \\ 0 & 0 & -1 \end{bmatrix}
   \end{gather*}

   This shows that the root system of $\Phi$ relative to $H = \text{span}_F\{h_\alpha,h_\beta\}$ is a rank two root system of type $A_2$. The Weyl group $\mathcal{W}$ associated with an $A_2$ root system is isomorphic to the symmetric group $S_3$ with order 6. \newline

   Now consider the subgroup of $Int(L)$, the group of inner automorphisms of $L$, generated by elements of the form:
   $$ \tau_{\alpha} = \exp{\text{ad } x_{\alpha}} \cdot \exp{\text{ad } -y_{\alpha}}\cdot \exp{\text{ad } x_{\alpha}}  $$

\iffalse
   Order the basis elements as such : $(x_{\alpha},h_{\alpha},y_{\alpha},x_{\beta},h_{\beta},y_{\beta},x_{\alpha + \beta},h_{\alpha + \beta},y_{\alpha +\beta})$ \newline

   Now calculating the adjoints explicitly:

   \begin{gather*}
     \text{ad }x_{\alpha} = \begin{bmatrix}
       0 & -2 & 0 & 0 &0 &0 &0 &0 \\
       0 & 0 & 1 & 0 &0 &0 &0 &0 \\
       0 & 0 & 0 & 0 &0 &0 &0 &0 \\
       0 & 0 & 0 & 0 &0 &0 &0 &0 \\
       0 & 0 & 0 & 0 &0 &0 &0 &1 \\
       0 & 0 & 0 & 1 &0 &0 &0 &0 \\
       0 & 0 & 0 & 0 &0 &0 &0 &0 \\
       0 & 0 & 0 & 0 &0 &0 &0 &0
     \end{bmatrix}
      \text{ad }y_{\alpha} = \begin{bmatrix}
       0 & 0 & 0 & 0 &0 &0 &0 &0 \\
       -1 & 0 & 0 & 0 &0 &0 &0 &0 \\
       0 & 2 & 0 & 0 &0 &0 &0 &0 \\
       0 & 0 & 0 & 0 &0 &1 &0 &0 \\
       0 & 0 & 0 & 0 &0 &0 &0 &0 \\
       0 & 0 & 0 & 0 &0 &0 &0 &0 \\
       0 & 0 & 0 & 0 &0 &0 &0 &0 \\
       0 & 0 & 0 & 0 &1 &0 &0 &0
     \end{bmatrix}
   \end{gather*}
\fi

   This subgroup contains at least the seven elements: $1,\tau_{\alpha},\tau_{\beta},\tau_{\alpha+\beta},\tau_{\alpha}^2,\tau_{\beta}^2,\tau_{\alpha+\beta}^2$, making this subgroup strictly larger than the Weyl group $\mathcal{W}$.
\end{proof}

\section*{Problem 3 (Humphreys 14.6)}
\begin{proof}
  Denote $\xi$ as a graph automorphism of the Coexter graph corresponding to the root system $\Phi$. Note that $\Phi$ is the irreducible root system corresponding to the simple Lie algebra $L$ and maximal toral subalgebra $H$. Now $\xi$ can be seen as a bijective map $\xi:\Delta \rightarrow \Delta$ in respect to an ordering of the simple roots in Cartan matrix associated with base $\Delta$. As a graph automorphism, it takes simple roots to simple roots in such a way that the Cartan matrix doesn't change. In other words, for an ordering $\{\alpha_1,...,\alpha_m\}$ with Cartan matrix $(\langle \alpha_i,\alpha_j \rangle)$, $\{\xi(\alpha_1),...,\xi(\alpha_m)\}$ is another ordering such that corresponding Cartan matrix $(\langle \xi(\alpha_i),\xi(\alpha_j) \rangle) = (\langle \alpha_i,\alpha_j \rangle)$. \newline

  The bijection extends uniquely to an automorphism $\psi_{\xi}: \Phi \rightarrow \Phi$ by Proposition (11.1). By invoking the isomorphism theorem, we extend again to an unique isomorphism $\Psi_{\xi}: L \rightarrow L$ extending the induced isomorphism $\pi_{\xi}: H \rightarrow H$. Since both extensions were determined uniquely, we can identify  $\xi \mapsto \Psi_{\xi} \in Aut(L)$ as a subgroup inside $Aut(L)$ isomorphic to the graph automorphisms of the Coexter graph of $\Phi$. The case for Dynkin diagrams follow similarly.
\end{proof}

\section*{Problem 4}
\begin{proof}
  Let $L$ denote a Lie algebra and ($\mathcal{U}(L),i$) its respective universal enveloping algebra with inclusion map $i: L \rightarrow \mathcal{U}(L)$. Given an (possibly countably infinite) ordered basis of $L$, $\{x_1,x_2,x_3,...\}$, let $$x_I = (\overline{x_{I(1)}})(\overline{x_{I(2)}})...(\overline{x_{I(m)}})= \pi(x_{I(1)} \otimes x_{I(2)} \otimes ... \otimes x_{I(m)}), \quad m = |I|$$ where $I \subset \mathbb{N}$ is a finite increasing sequence and $\pi:\mathcal{I}(L) \rightarrow \mathcal{U}(L)$ is the canonical map from the tensor algebra of $L$ to the universal enveloping algebra of $L$. \newline

  Let $y \in \mathcal{U}(L)$. Since the elements of a PBW basis (17.3) form a basis of $\mathcal{U}(L)$, $$y = \sum_I c_I x_I, \quad c_I \in F$$ is summed over all such finite increasing sequences $I$. All but a finite number of $c_I$ will be nonzero, so this sum will be finite. \newline

  We first identify finite-dimensional $\mathcal{U}(L)$-modules $V$ as a $L$-modules as follows: given an action on $V$ of the form
  $ v \mapsto y \cdot v $ for some $y \in \mathcal{U}(L)$, we use the form above to see the action as
  $$ v \mapsto (\sum_I c_I x_I)\cdot v = \sum_I [(c_Ix_I)\cdot v] = \sum_I c_I(x_I \cdot v)$$
  By our definition for $x_I$
  $$ x_I \cdot v = ((\overline{x_{I(1)}})(\overline{x_{I(2)}})...(\overline{x_{I(m)}}))\cdot v $$

  Now using the action above, define an $L$-module structure on $V$ as follows:
  for any element $\overline{x_{j}} \in \mathcal{U}(L)$, $\pi^{-1}(\overline{x_{j}}) = i^{-1}(\overline{x_{j}}) = x_j$ as $i$ is the restriction of the $\pi$ to $L$. Furthermore, this inverse map is well-defined as the PBW theorem shows that $i$ is injective and hence isomorphic onto its image $\pi(L)$.
  %Define
  %$$y_I = (\pi^{-1}(\overline{x_{I(1)}}))(\pi^{-1}(\overline{x_{I(2)}}))...(\pi^{-1}(\overline{x_{I(m)}})) = (i^{-1}(\overline{x_{I(1)}}))(i^{-1}(\overline{x_{I(2)}}))...(i^{-%1}(\overline{x_{I(m)}}))$$
  %as $i$ is defined as the restriction of $\pi$ to $L$. This map is well-defined as the PBW theorem shows that $i$ is injective and hence isomorphic onto its image $\pi(L)$. %Thus,
  %$$ y_I = x_{I(1)}x_{I(2)}...x_{I(m)}  $$
  Hence, the action of $x_j$ on $V$ will be
 $$ v \mapsto \overline{x_{j}} \cdot v $$ under the identification $x_j \mapsto \overline{x_{j}}$. Since $x_j$ was an arbitrary basis element of $L$, we can extend this action linearly to all of $L$, giving us an action from the $\mathcal{U}(L)$-module structure on $V$. \newline

 Conversely, given a $L$-module structure on finite-dimensional vector space $V$, we can extend this to an $\mathcal{U}(L)$-action as $i(L)$ and $1$ generate $\mathcal{U}(L)$ by the PBW Theorem (Corollary C of 17.3). Thus, it suffices to define the action on the image of basis elements of $L$: define the action of $\overline{x_j} = i(x_j) \in \mathcal{U}(L)$ by $L$-action for basis element $x_j$
 $$ v \mapsto x_j \cdot v $$

Let ${\bf FinRep}_{\mathcal{U}(L)}$ be the category of finite-dimensional $\mathcal{U}(L)$-representations and ${\bf FinRep}_{L}$ be the category of finite-dimensional $L$-representations. Define the functors $F:{\bf FinRep}_{\mathcal{U}(L)} \rightarrow {\bf FinRep}_{L}$ and $G:{\bf FinRep}_{L} \rightarrow  {\bf FinRep}_{\mathcal{U}(L)}$ by the identification schemes above.
\newline

We see that for $x \in \text{Obj}({\bf FinRep}_{L})$, $FG(x) = x$ and for $y \in \text{Obj}({\bf FinRep}_{\mathcal{U}(L)})$, $GF(y) = y$. Similarly, morphisms between ${\mathcal{U}(L)}$-representations define morphisms between $L$-representations which then linearly extend again to the same morphisms between ${\mathcal{U}(L)}$-representations by the properties of the PBW basis. Also, morphism between $L$-representations linearly extend to morphisms between ${\mathcal{U}(L)}$-representations which define the same morphisms between between $L$-representations. Hence, $FG = 1_{{\bf FinRep}_{L}}$ and $GF = 1_{{\bf FinRep}_{\mathcal{U}(L)}}$, showing us equivalence of categories.
\end{proof}

\end{document}
