\documentclass[12pt]{article}%
\usepackage{amsfonts}
\usepackage{amsmath}
\usepackage[a4paper, top=2.5cm, bottom=2.5cm, left=2.2cm, right=2.2cm]%
{geometry}
\usepackage{times}
\usepackage{amsmath}
\usepackage{amssymb}
\newenvironment{proof}[1][Proof]{\textbf{#1.} }{\ \rule{0.5em}{0.5em}}

\begin{document}

\title{Math 774 Homework 4}
\author{Edward Kim}
\date{\today}
\maketitle

\section*{Problem 1 (Humphreys 10.1)}
\begin{proof}
  From Problem 9.2, we know that $\Phi^{\vee}$ is a root system. By Theorem 10.1 (pg 48), we know that every base of $\Phi^{\vee}$ will be of the form $\Delta^{\vee}(\gamma)$, the set of indecomposible roots in $(\Phi^{\vee})^{+}(\gamma)$, where
  $\gamma \in E$ is a regular root such that $(\gamma,\alpha^{\vee}) > 0$ for all $\alpha^{\vee} \in \Delta^{\vee}$. Thus, it suffices to prove that $\Delta^{\vee} = \Delta^{\vee}(\gamma)$. However, note that $(\gamma,\alpha^{\vee}) = \frac{2}{(\alpha,\alpha)}(\gamma,\alpha)$ and $(\alpha,\alpha) > 0$. Hence, if $(\gamma, \alpha) > 0$, then $(\gamma,\alpha^{\vee}) > 0$ as well. Furthermore, if a root $\beta \in \Phi$ is indecomposable then $\beta^{\vee} \in \Phi^{\vee}$ must also be indecomposable as simply scaling the roots will not yield decomposability. We now see that $\Delta^{\vee}(\gamma)$ will precisely be the positive roots in $\Delta(\gamma)$ scaled by their corresponding factors. As $\Delta = \Delta(\gamma)$, by definition of $\Delta^{\vee}$, it follows that $\Delta^{\vee} = \Delta^{\vee}(\gamma)$ as required.
\end{proof}

\section*{Problem 2 (Humphreys 10.2)}
\begin{proof}
  We first consider the set $S = (\mathbb{Z}\alpha + \mathbb{Z}\beta) \cap \Phi$. Since $\alpha,\beta \in \Delta$ are two roots in $\Phi$, Table 1 and Figure 1 of Section 9.3 classifies all possible root systems which can be generated, depending on the obtuse angle between $\alpha,\beta$. Furthermore, all roots generated with $\alpha,\beta$ by rotation and reflections must also be roots in $\Phi$ (See Problem 9.4). Thus, $S$ will be a rank two root system in the subspace spanned by $\alpha,\beta$ isomorphic to one of $A_1 \times A_1, A_2, B_2, G_2$.
  \newline

  \noindent Now consider an arbitrary subset $\Delta' \subseteq \Delta$ and define $S' = (\sum_{\alpha' \in \Delta'} \mathbb{Z}\alpha') \cap \Phi$. We will directly check the root system axioms for $S'$. Since $S' \subseteq \Phi$, Axioms $(R1),(R2),(R4 )$ immediately follow for $S'$. We just check that every $\sigma_{\psi}$ leaves $S'$ invariant for all $\psi \in S'$.
  By definition, for any $\beta \in S'$, $\beta = \sum_{\alpha \in \Delta} k_{\alpha}\alpha$ where either all $k_{\alpha} \geq 0$ or $k_{\alpha} \leq 0$.
  However, by definition of $S'$, $k_{\gamma} = 0$ for $\gamma \in \Delta / \Delta'$. Recall that $\sigma_{\psi}(\beta) = \beta - \langle \beta,\psi \rangle \psi$. In light of the observation above, if $\psi,\beta \in S'$, then $\sigma_{\psi}(\beta) \in S'$ since the subtraction and scaling of two roots in $S'$ will also be contained in $S'$. We have shown that every reflection map $\sigma_{\psi}$ leaves $S'$ invariant, and that $S'$ is a root system of rank $|\Delta'|$ which spans the subspace of $E$ generated by the $\alpha' \in \Delta'$.
\end{proof}

\section*{Problem 3 (Humphreys 10.4)}
\begin{proof}
We directly calculate the roots in $\Phi^{+}$ to be $\alpha,\beta,\alpha + \beta,\beta + 2\alpha,\beta + 3\alpha, 2\beta + 3\alpha$. Now:
\begin{gather*}
  \beta + 2\alpha = (\beta + \alpha) + \alpha \\
  \beta + 3 \alpha = ((\beta + \alpha) + \alpha) + \alpha \\
  2\beta + 3\alpha = (((\beta + \alpha) + \alpha) + \alpha) + \beta
\end{gather*}
  Each of the partial sums enclosed by parathesis is a root as shown by its predeccesor and each of the terms is a simple root. This directly shows that the lemma holds for $G_2$.
\end{proof}

\section*{Problem 4 (Humphreys 10.6)}
\begin{proof}
Define the function $sn: \mathcal{W} \rightarrow \{\pm 1\}$ by $\sigma \mapsto (-1)^{l(\sigma)}$ where $l:\mathcal{W} \rightarrow \mathbb{Z}_{\geq 0}$ is
the length function relative to a base $\Delta$ as defined in Section 10.3. \newline

Let $\sigma_{\alpha},\sigma_{\beta}$ be the reflection maps of $\alpha,\beta \in \Phi$ and let $\sigma_{\alpha}\sigma_{\beta} = \sigma_{\alpha_1} ... \sigma_{\alpha_n}\sigma_{\beta_1}_... \sigma_{\beta_m}$ be the concatentation of their expansion into minimal simple reflections. Reindex this string by $\sigma_{\tau_1},...,\sigma_{\tau_{n+m}}$ such that $\tau_1 = \alpha_{1},\tau_2 = \alpha_2,...,\tau_n = \alpha_n, \tau_{n+1} = \beta_1,...,\tau_{n+m} = \beta_m$. Finally let $\theta_i = \sigma_{\tau_{1}}...\sigma_{\tau_{i-1}}(\tau_{i}), 1 \leq i \leq n+m$. \newline

We start with the case when $\theta_i \neq \theta_j$ for any $1\leq i < j \leq n+m$. In other words,
there does not exist a root $\gamma \in \Phi^+$ such that $\tau_j = \sigma_{\tau_{j+1}}...\sigma_{\tau_{n+m}}(\gamma)$ and $\tau_i = \sigma_{\tau_{i+1}}... \sigma_{\tau_{j-1}}(\tau_j) = \sigma_{\tau_{i+1}}...
\sigma_{\tau_{n+m}}(\gamma)$. This shows that $l(\sigma_{\alpha}\sigma_{\beta}) = l(\sigma_{\alpha}) + l(\sigma_{\beta})$ as the concatentation cannot be reduced further by Lemma C of Section 10.2. So the following holds:
 $$sn(\sigma_{\alpha}\sigma_{\beta}) = (-1)^{l(\sigma_{\alpha}\sigma_{\beta})} = (-1)^{l(\sigma_{\alpha}) + l(\sigma_{\beta})} = (-1)^{l(\sigma_{\alpha})} + (-1)^{l(\sigma_{\beta})} = sn(\sigma_{\alpha}) + sn(\sigma_{\beta})$$

Now consider the case where there exists $i < j$ such that $\theta_i = \theta_j$. We invoke Lemma C of Section 10.2 to yield that:
$$ \sigma_{\tau_1}...\sigma_{\tau_j} =  \sigma_{\tau_1}... \sigma_{\tau_{i-1}}\sigma_{\tau_{i+1}}... \sigma_{\tau_{j-1}} $$
We can continue to reduce $\sigma_{\alpha}\sigma_{\beta}$ in this fashion until completely reduced to its minimal form. As we remove two factors at step of the reduction,
$$ sn(\sigma_{\alpha}\sigma_{\beta}) = (-1)^{l(\sigma_{\alpha}) + l(\sigma_{\beta}) - 2n} = (-1)^{l(\sigma_{\alpha}) + l(\sigma_{\beta})} $$ for some number of steps $n$. Both cases show that $sn$ is indeed a homomorphism.
\end{proof}

\section*{Problem 5 (Humphreys 10.9)}
\begin{proof}
  By Theorem 10.3, we know that the Weyl group $\mathcal{W}$ acts simply transitively on the Weyl chambers of $\Phi$ i.e for any $\mathfrak{C}_1,\mathfrak{C}_2$ Weyl chambers of $\Phi$, there exists unique $\sigma \in \mathcal{W}$ such that $\sigma(\mathfrak{C}_1) = \mathfrak{C}_2$. Applying this property to the chambers $\Phi^{+}$ and $\Phi^{-}$, we see that there exists unique $\sigma \in \mathcal{W}$ taking $\Phi^{+}$ to $\Phi^{-}$. To see that $\sigma$ must involve $\sigma_{\alpha}$ for all $\alpha \in \Delta$, note that each $\sigma_{\alpha}$ permute all of roots in $\Phi^+$ except for $\alpha$. Consider $\delta = \frac{1}{2} \sum_{\beta > 0} \beta$. By the corollary of Lemma B of Section 10.2, $\sigma_{\alpha}(\delta) = \delta - \alpha$. Thus if $\sigma$ does not contain $\sigma_{\alpha}$ for some $\alpha \in \Delta$, then $\sigma(\delta) \not\in \Phi^-$ since $\alpha$ is a term with positive coefficient one in $\sigma(\delta)$, contradicting its definition.
  %This argument suggests that $l(\sigma) = |\Phi^{+}|$
  %Consider an simple root $\beta \in \Delta$ such that if $\sigma = \sigma_{\alpha_1}...\sigma_{\alpha_n}$ is the reduced form of $\sigma$,
\end{proof}

\section*{Problem 6 (Humphreys 10.12)}
\begin{proof}
  By Theorem 10.3 (e), it suffices to show that $\sigma(\Delta) = \Delta$. However, $\Delta = \Delta(\lambda)$ where $\lambda \in \mathfrak{C}(\Delta)$. Thus, we will prove that $\sigma(\Delta(\lambda)) = \Delta(\lambda)$.
  Since $\sigma(\lambda) = \lambda$, if we can show that $\sigma(\Delta(\lambda)) = \Delta(\sigma(\lambda))$, we will be done. But, by Lemma 9.2, $\langle \sigma(\lambda),\sigma(\beta) \rangle = \langle \lambda, \beta \rangle$ for any root $\beta \in \Phi$. This means that $\sigma(\gamma)$ will precisely be the regular element lying in the positive halfspaces of the roots $\sigma(\alpha)$. Thus, the indecomposable elements of $\Phi^+({\sigma(\gamma)})$ will be $\sigma(\Delta(\gamma))$, giving the required result.
\end{proof}

\section*{Problem 7 (Humphreys 11.4)}
\begin{proof}
  Define the map $\phi: \mathcal{W} \rightarrow \mathcal{W}_1 \times ... \times \mathcal{W}_n$ to be $\sigma \mapsto (\prod_{i \in \Delta_1} \sigma_i,...,\prod_{i \in \Delta_n} \sigma_i)$ which divides the terms of its simple reflection expansion of $\sigma$ according to its respective irreducible components; $\sigma_{\alpha_i}$ maps to the $i^{th}$ component if $\alpha_i \in \Delta_i$. The map also respects the ordering of the terms found in the expansion. We will first show that this map is well-defined. Begin by noting that if $\alpha,\beta \in \Delta$ such that $\alpha \in \Delta_i, \beta \in \Delta_j$ are both contained in mutually orthogonal subsets of $\Delta$, then $\sigma_{\alpha}(\beta) = \beta - \langle \beta,\alpha \rangle \alpha = \beta$. Thus, simple reflections fix simple roots from other orthogonal subsets. Let $\sigma = \sigma_{\alpha_1}...\sigma_{\alpha_n}, \alpha_i \in \Delta$ be its simple reflection expansion. By our observation above, if there exists some index $1 \leq s < n$ such that $\sigma_{\alpha_1}...\sigma_{\alpha_n} = \sigma_{\alpha_1}...\sigma_{\alpha_{s-1}}\sigma_{\alpha_{s+1}}...\sigma_{\alpha_{n-1}}$, then $\alpha_s,\alpha_n$ must be contained in the base of the same irreducible component. Hence, the map above does not depend on the simple reflection decomposition of $\sigma$ as reducibility in $\sigma$ translates to reducibility of a string found in a component in the direct product, namely $\sigma_{\alpha_{i_1}}...\sigma_{\alpha_{i_r}}$ for $\alpha_{i_1},...,\alpha_{i_r} \in \Delta_i$.
  To show that $\phi$ is a group homomorphism, note that the argument above for the well-definedness of $\phi$ also shows that:
  $$ \phi(\sigma\gamma) = ((\prod_{i \in \Delta_1} \sigma_i)(\prod_{i \in \Delta_1} \gamma_i),...,(\prod_{i \in \Delta_n} \sigma_i)(\prod_{i \in \Delta_n} \gamma_i)) $$ for $\sigma,\gamma \in \mathcal{W}$.
  $(\prod_{i \in \Delta_j} \gamma_i)$ represents the ordered string of simple reflection maps corresponding to elements in $\Delta_j$ found in the expansion of $\gamma$.
  Thus,
  $$ \phi(\sigma\gamma) = \phi(\prod_{i \in \Delta_1} \sigma_i,...,\prod_{i \in \Delta_n} \sigma_i)(\prod_{i \in \Delta_1} \gamma_i,...,\prod_{i \in \Delta_n} \gamma_i) = \phi(\sigma)\phi(\gamma)$$

  Surjectivity follows from the observation that one can concatentate all of the components to create an element in $\mathcal{W}$. Injectivity follows from a similar argument by concatentating simple reflections which all are equal to the identity map on the $\mathcal{W}_i$. As all the simple roots in the $\Delta_i$ partitions are mutually orthogonal (their reflection maps fix simple roots in other partitions), any concatentation of the simple reflections equal to the identity map of their respective components will be equal to the identity map on the entirety of $\mathcal{W}$. Therefore, $\phi$ is an isomorphism.
\end{proof}
\end{document}
